\documentclass[a4paper]{article}

\usepackage{color}
\usepackage{url}
\usepackage[T2A]{fontenc} % enable Cyrillic fonts
\usepackage[utf8]{inputenc} % make weird characters work
\usepackage{graphicx}

\usepackage[english,serbian]{babel}

\usepackage[unicode]{hyperref}
\hypersetup{colorlinks,citecolor=green,filecolor=green,linkcolor=blue,urlcolor=blue}

\newtheorem{primer}{Primer}[section]

\begin{document}

\title{DALL E 2 - OpenAI\\ \small{Seminarski rad u okviru kursa\\Tehničko i naučno pisanje\\ Matematički fakultet}}

\author{Ognjen Marković\\ ognjen.mark03@gmail.com\\ Matija Milovanović\\ matija.milovanovic55@gmail.com\\ Bogdan Milovanović\\ email\\ Bojana Radovanović\\ bojanaradovanovic080@gmail.com }
\date{11.~novembar 2022.}
\maketitle

\abstract{
DALL-E I DALL-E 2 su modeli mašinskog učenja koje je razvio OpenAI za generisanje digitalnih
slika pomoću opisa koje dobija od NLP - a, koji se nazivaju „ prompts “. DALL-E je prvi put
spomenut na blogu OpenAi-a u januaru 2021. i tada je koristio verziju GPT-3 modifikovanu za
generisanje slika. U aprilu 2022. OpenAI je najavio DALL-E 2, naslednika dizajniranog da
generiše realističnije slike u višim rezolucijama koje mogu kombinovati koncepte, atribute i
stilove. Naziv softvera je sastavni deo imena animiranog robota Pikar lika VALL-E i španskog
nadrealističkog umetnika Salvadora Dalija.}

\tableofcontents

\newpage

\section{Uvod}
\label{sec:uvod}
OpenAI je istraživačka laboratorija za veštačku inteligenciju (AI) koju čine profitna korporacija OpenAI LP I njena matična kompanija, neprofitna OpenAI Inc. Kompanija, koja se smatra konkurentom DeepMindu, sprovodi istraživanja u oblasti veštačke inteligencije sa navedenim ciljem promovisanja i razvoja prijateljske veštačke inteligencije na način koji koristi čovečanstvu u celini. Organizaciju su osnovali u San Francisku krajem 2015. Ilon Mask, Sem Altman I drugi, koji su zajednički obecá li milijardu dolara. Mask je podneo ostavku iz odbora u februaru 2018, ali je ostao donator. U 2019, OpenAI LP je od Microsofta dobio investiciju od milijardu dolara. Neki naučnici, poput Stivena Hokinga i Stjuarta Rasela, izrazili su zabrinutost da ako napredna veštačka inteligencija jednog dana dobije sposobnost da se redizajnira sve vecó m brzinom, nezaustavljiva „eksplozija inteligencije“ može dovesti do izumiranja ljudi. Mask karakteriše veštačku inteligenciju kao „najvecú egzistencijalnu pretnju čovečanstva“. Osnivači OpenAI su je strukturisali kao neprofitnu organizaciju kako bi mogli da fokusiraju svoje istraživanje na stvaranjepozitivnog dugoročnog uticaja na ljude.



\section{DALL-E 2}
\label{DALLE2}
\subsection{Tehnologija}
\label{subsec:tehnologija}

Model generativnog unapred obučenog transformatora (GPT) je prvobitno razvio OpenAI 2018. godine, koristecí arhitekturu transformatora. Prva iteracija, GPT, je povecá na da bi proizvela GPT-2 2019. godine; 2020. ponovo je povecá na da bi proizvela GPT-3, sa 175 milijardi parametara. DALL-E-ov model je multimodalna implementacija GPT-3 sa 12 milijardi parametara koji „zamenjuje tekst za piksele“, obučen na parovima tekst-slika sa Interneta. DALL-E 2 koristi 3,5 milijardi parametara, što je manji broj od svog prethodnika.



\begin{table}[h!]
\begin{center}
\caption{U tabeli \ref{tab:tabela1} je prikazan hronološki razvitak DALL-E 2.} 
\vspace{0.5cm}
\begin{tabular}{|c|c|c|c|c|} \hline
2018&2019&2020&2021&2022\\ \hline
GPT&GPT-2&GPT-3 &DALL-E&DALL-E 2\\ \hline
\end{tabular}
\label{tab:tabela1}
\end{center}
\end{table}

DALL-E je razvijen i objavljen javnosti u saradnji sa CLIP-om (Pre-trening za kontrastni jezik). CLIP je poseban model zasnovan na učenju nulte slike koji je obučen na 400 miliona parova slika sa tekstualnim natpisima prebačenim sa interneta. Njegova uloga je da „razume i rangira“ DALL-E-ov izlaz predviđanjem koji je naslov sa liste od 32.768 naslova nasumično odabranih iz skupa podataka (od kojih je jedan bio tačan odgovor) najprikladniji za sliku. Ovaj model se koristi za filtriranje vecé početne liste slika koje generiše DALL-E da bi se odabrali najprikladniji rezultati. DALL-E 2 koristi model difuzije uslovljen ugrađivanjem CLIP slika, koje se, tokom zaključivanja, generišu iz CLIP ugrađivanja teksta od strane prethodnog modela. Kad bi smo pojednostavili stvari, ovaj model funkcioniše tako što počinje sa slikom koja se sastoji od nasumičnih piksela („šum“) i postepeno „uklanja šum“ slike. Tokom procesa uklanjanja
šuma, on se vodi ka slici koja odgovara početnom izvornom promptu.

\end{document}

\subsection{Mogućnosti}
\label{subsec:mogućnosti}

\textbf{DALL-E 2 može da:}
\begin{itemize}
\item generiše slike u više stilova, uključujucí fotorealistične slike, slike I emodžije
\item da manipuliše i preuređuje objekte na svojim slikama
\item ispravno postavi elemente dizajna u nove kompozicije bez eksplicitnih instrukcija
\end{itemize} 

Thom Dunn koji je pisao za BoingBoing je primetio da „Na primer, kada se zamoli da nacrta daikon rotkvu kakoduva nos, pijucka kafu ili se vozi monociklom, DALL-E često crta maramicu, ruke i stopala na uverljivim mestima.“DALL-E je pokazao sposobnost da „popuni prazna mesta“ kako bi zaključio odgovarajucé detalje bez specifičnihnapomena kao što je dodavanje božićnih slika uputstvima koja se obično povezuju sa proslavom, i odgovarajucépostavljene senke slikama koje ih ne pominju. Šta više, DALL-E pokazuje široko razumevanje vizuelnih idizajnerskih trendova.

DALL-E je u stanju da proizvede slike za širok spektar proizvoljnih opisa sa različitih gledišta samo saretkim greškama. Mark Ridl, vanredni professor na Tehničkoj školi za interaktivno računarstvo uDžordžiji, otkrio je da DALL-E može da kombinuje koncepte (opisan kao ključni element ljudskekreativnosti) i ima sposobnost vizuelnog rasuđivanja dovoljnu da reši Rejvenove matrice (vizuelni testovikoji se često primenjuju ljudima za merenje inteligencije).

DALL-E 2 može da kreira veoma realistične slike kao npr. prikazan \ref{fig:pas} portret baseta (slika 1.).

\begin{figure}[h!]
\begin{center}
\includegraphics[scale=0.75]{pas.jpg}
\end{center}
\end{figure}

Ali trik u svemu ovome je što ovaj pas ne postoji.. Ova slika psa nije nastala tako što je osoba slikala svogčetvoronoćnog prijatelja, već je slika koju je napravilo DALL-E 2. Takođe, u stanju je da kreira veoma realistične slike ljudi npr. prikaz \ref{fig:zena}  realne fotografije mlade žene plavih očiju i plave kose (slika 2.) ,kao i objekte koji u stvarnosti ne postoje.
